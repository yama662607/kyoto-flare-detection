%%%%%%%%%%%%%%%%%%%%%%%%%%%%%%%%%%%%%%%%%%%%%%%%%%%%%%%%%%%%%%%%%%%%%%%%%
%%% PASJ LaTeX template for draft(body) <2024/07/01>
%%%
%%% IMPORTANT NOTICE FOR AUTHORS
%%%  1. Do NOT use \def/\renewcommand.
%%%  2. Do NOT redefine commands provided by PASJ02.cls.
%%%  3. LETTER article must NOT exceed ``six pages'' in length in PASJ's publication layout format.
%%%    Do NOT change the default font setting of pasj02.cls to avoid obtaining an inaccurate page estimation.
%%%  4. ``\draft'' creates single column and double spaces format.
%%%
%%% Instructions to authors: https://academic.oup.com/pasj/pages/General_Instructions
%%% Author's guide (in Japanese): https://www.asj.or.jp/pasj/guide/
%%%%%%%%%%%%%%%%%%%%%%%%%%%%%%%%%%%%%%%%%%%%%%%%%%%%%%%%%%%%%%%%%%%%%%%%%
\documentclass[]{pasj02}
%\draft
\usepackage[switch,mathlines]{lineno} % add line number to manuscript
\usepackage{comment}
\usepackage{url}
%\usepackage{natbib}


\jyear{2026}
\Received{}%{yyyy/mm/dd}
\Accepted{}%{yyyy/mm/dd}
%\Published{yyyy/mm/dd}

%\graphicspath{{./}{figures/}}


\begin{document}

\title{Time-Resolved Connection between Starspots and Flares in Nearby Young Solar-type Stars Observed by TESS}

%%% begin:list of authors
% Do NOT capitalize all letters in "textsc".
\author{
 Daijiro \textsc{Hitotsuyanagi},\altaffilmark{1}\altemailmark \email{hitotsuyanagi.daijiro.23f@st.kyoto-u.ac.jp}
 Hiroto \textsc{Yamada},\altaffilmark{1}
 %\altemailmark %\email{yamada.hiroto.74h@st.kyoto-u.ac.jp}
 Daisuke \textsc{Yamashiki},\altaffilmark{1}
 %\altemailmark %\email{yamashiki.daisuke.24f@st.kyoto-u.ac.jp}
 Kazuma \textsc{Ishihara},\altaffilmark{1}
 %\altemailmark %\email{k.ishihara@kusastro.kyoto-u.ac.jp}
 Yosuke \textsc{Yamashiki},\altaffilmark{2}
 %\altemailmark %\email{yamashiki.yosuke.3u@kyoto-u.ac.jp}
 and
 Kosuke \textsc{Namekata}\altaffilmark{3,4,5,6}%\altemailmark
 \orcid{0000-0002-1297-9485} %\email{namekata@kusastro.kyoto-uac.jp}
}
\altaffiltext{1}{Department of Physics, Faculty of Science, Kyoto University, Kitashirakawa-Oiwake-cho, Sakyo-ku, Kyoto 606-8502, Japan}
\altaffiltext{2}{Graduate School of Advanced Integrated Studies in Human Survivability (GSAIS), Kyoto University, Yoshida-Nakaadachi-cho, Sakyo-ku, Kyoto 606-8306, Japan}
%\altaffiltext{3}{Unit of Synergetic Studies for Space, Kyoto University, Yoshida-Honmachi, Sakyo-ku, Kyoto 606-8501, Japan}
\altaffiltext{3}{Heliophysics Science Division, NASA Goddard Space Flight Center, 8800 Greenbelt Road, Greenbelt, MD 20771, USA}
\altaffiltext{4}{The Catholic University of America, 620 Michigan Avenue, N.E. Washington, DC 20064, USA}
\altaffiltext{5}{The Hakubi Center for Advanced Research, Kyoto University, Yoshida-Honmachi, Sakyo-ku, Kyoto 606-8501, Japan}
\altaffiltext{6}{Department of Physics, Kyoto University, Kitashirakawa-Oiwake-cho, Sakyo-ku, Kyoto, 606-8502, Japan}


%\footnotetext[$\dag$]{Present address: ....}

%%% end:list of authors

%% !!! Select 3 to 5 words from PASJ's key words !!!
%% List of Key Words: https://academic.oup.com/pasj/pages/Pasj_Keywords
%% "\KeyWords{ }" always has to be placed before ``\maketitle''
\KeyWords{stars: activity --- stars: flare --- stars: late-type --- stars: solar-type --- starspots --- Sun: flares --- sunspots}

\maketitle

\begin{abstract}
Superflares are energetic explosions on stellar surface with energies of $10^{33}$--$10^{36}$ erg, significantly exceeding those of typical solar flares. While previous studies have suggested that these events are driven by magnetic energy stored in large starspots, the detailed time-resolved relationship between starspot area and flare activity on individual stars has remained unclear. In this paper, we investigate the time evolution of magnetic activity on three representative young solar-type stars (EK Draconis, DS Tucanae A, and V889 Herculis) using $\sim$7 years of photometric data from the Transiting Exoplanet Survey Satellite (TESS).
We automatically detected stellar flares and derived the flare frequency, starspot area, and rotational period for each TESS sector covering $\sim$27 days.
As a result, we found that the flare frequency and starspot area vary significantly across sectors, although we could not identify any activity-cycle-like pattern.
There is a positive correlation between the starspot area and flare occurrence frequency for all three targets and the power-law dependence is consistent among the stars.
This result supports the physical picture that superflares on young solar-type stars are powered by magnetic energy stored in large starspots, analogous to solar flares, and that the energy release rate changes as the total stored magnetic energy varies.
Furthermore, from the analysis of EK Draconis, we find a possible dependence of starspot area on rotation period, which may suggest that large starspots preferentially form at mid-latitudes. These findings demonstrate that the magnetic activity mechanisms established for the Sun extend to the extreme magnetic activity observed on young active stars.

\end{abstract}

\pagewiselinenumbers

\section{Introduction}

\begin{table*}[!tp]
\caption{Stellar parameters.\label{tab:targets}}
\centering
\begin{tabular}{lccc}
\hline
Parameter & V889~Her (TIC~471000657) & DS~Tuc (TIC~410214986) & EK~Dra (TIC~159613900)\\
\hline
Spectral Type &
G0V$^{a}$ &
G6V$\pm1$/K3V$\pm1^{f}$ &
G1.5V$^{c}$ \\

$V_{\rm mag}$ &
$7.45\pm0.04^{a}$ &
$8.55\pm0.01/9.65\pm0.03^{f}$ &
$7.60\pm0.01^{d}$ \\

Age (Myr) &
$30^{a}$ &
$45\pm4^{g}$ &
50--125$^{c,h}$ \\

$T_{\rm eff}$ (K) &
$5830\pm50^{a}$ &
$5430\pm80/4700\pm90^{f}$ &
5560--5750$^{c,e,i}$ \\

Radius ($R_\odot$) &
$1.09\pm0.05^{a}$ &
$0.96\pm0.03/0.86\pm0.04^{f}$ &
$0.94\pm0.07^{c}$ \\

Mass ($M_\odot$) &
$1.06\pm0.02^{a}$ &
$1.01\pm0.06/0.84\pm0.06^{f}$ &
$0.95\pm0.04^{c}$ \\

Distance (pc) &
$35.36\pm0.02^{b}$ &
$44.78\pm0.03^{b}$ &
$34.40\pm0.03^{b}$ \\

$P_{\rm rot}$ (d) &
$1.3371\pm0.0002^{a}$ &
$2.85^{+0.04}_{-0.05}/-^{f}$ &
$2.766\pm0.002^{c}$ \\

$v\sin i$ (km\,s$^{-1}$) &
$39.0\pm0.5^{a}$ &
$17.8\pm0.2/14.4\pm0.3^{f}$ &
$16.4\pm0.1^{c}$ \\

Inclination (deg) &
$\approx55^{a}$ &
$>82^\circ/-^{f}$ &
$60\pm5^{c}$ \\

Binarity &
single$^{a}$ &
binary / exoplanet system &
low-mass companion$^{c}$ \\

\hline
\end{tabular}

\vspace{1mm}
\begin{minipage}{0.9\textwidth}
\footnotesize
$^{a}$ Table~2 of \citet{2003A&A...411..595S}. $^{b}$ Gaia EDR3 \citep{2021A&A...649A...1G}. $^{c}$ \citet{2017MNRAS.465.2076W}. $^{d}$ \citet{2000A&A...355L..27H}. $^{e}$ \citet{2018A&A...620A.162J}. $^{f}$ \citet{2019ApJ...880L..17N}. $^{g}$ \citet{2015MNRAS.454..593B}. $^{h}$ Reported ages for EK~Dra range from 30--125 Myr depending on the study. $^{i}$ For flare-energy calculations we adopt $T_{\rm eff}\approx5700$\,K from \citet{2005A&A...435..215K}.
\end{minipage}

\end{table*}

Solar flares are energetic explosions on the solar surface caused by the sudden release of magnetic energy stored around sunspots via magnetic reconnection \citep{2011LRSP....8....6S,2010ARA&A..48..241B,2024LRSP...21....1K}.
While typical solar flares have energies of $10^{29}$--$10^{32}$ erg \cite{2012ApJ...759...71E,2022LRSP...19....2C}, the Kepler mission has revealed that solar-type stars (G-type main-sequence stars) can produce ``superflares'' with energies ranging from $10^{33}$ to $10^{36}$ erg (e.g., \cite{2012Natur.485..478M,2013ApJS..209....5S,2016ApJ...829...23D,2021ApJ...906...72O,2022PASJ...74.1295Y,2024Sci...386.1301V}).
Subsequent studies have suggested that these superflares are also driven by magnetic energy stored in large starspots (e.g., \cite{2013PASJ...65...49S,2013ApJ...771..127N,2015PASJ...67...33N,2019ApJ...876...58N,2021ApJ...906...72O,2025ApJ...985..158T}), similar to solar flares (e.g., \cite{2000ApJ...540..583S,2019LRSP...16....3T}).
Understanding the occurrence frequency and properties of these superflares is crucial not only for stellar astrophysics but also for evaluating their potential impact on planetary habitability (e.g., \cite{2010AsBio..10..751S,2016NatGe...9..452A,2017MNRAS.465L..34A,2020IJAsB..19..136A,2025kiss.rept.....L}).

Previous statistical studies using Kepler data have shown that the flare frequency increases with the starspot area (e.g., \cite{2013PASJ...65...49S,2013ApJ...771..127N,2019ApJ...876...58N,2021ApJ...906...72O}).
However, most of these studies were ensemble analyses averaging over many stars, or analyses of individual stars over relatively short observational baselines.
Consequently, it remains unclear whether the flare frequency of a single star changes in response to the time evolution of its starspot area (cf.  \cite{2021ApJ...922L..23A}, Supplementary Information in \cite{2022NatAs...6..241N}).
While it is well established for the Sun that flare activity correlates with the 11-year sunspot cycle (cf. \cite{2019LRSP...16....3T}), verifying this ``time-resolved'' relationship on other active stars has been not explored.

The Transiting Exoplanet Survey Satellite (TESS) mission \citep{2015JATIS...1a4003R} provides a unique opportunity to address this issue.
TESS has been performing an all-sky survey since 2018, accumulating photometric data over a baseline of approximately 7 years to date.
This long-term baseline enables us to investigate the time evolution of magnetic activity on individual stars.
In particular, the TESS mission duration now exceeds Kepler’s four-year mission and therefore offers an advantage for investigating longer-timescale variations, including activity cycles, that could not be explored during the Kepler era.



In this study, we investigate the time-resolved relationship between starspot area and flare frequency on young solar-type stars.
We utilized the long-term optical photometric data from TESS to analyze three representative young solar-type stars: EK Draconis, DS Tucanae A, and V889 Herculis.
Young solar-type stars are of particular interest as they serve as proxies for the ``young Sun,'' offering insights into the magnetic environment of the early solar system.
Also, young solar-type stars are know to be very flare active, so they are suitable
These three stars are known as bright young solar-type stars and stellar parameters are well known as in Table \ref{tab:targets}, their flare properties and starspot properties are well characterized so far \citep{2005LRSP....2....8B,2005A&A...435..215K,2007LRSP....4....3G,2017MNRAS.465.2076W,2018A&A...620A.162J,2022A&A...661A.148C,2022NatAs...6..241N,2022ApJ...926L...5N,2025ApJ...993...80N,2025arXiv251203830G}.
In this study, by deriving the starspot area and flare occurance rate for each observational sector, we specifically examine whether the correlation observed in statistical ensemble studies holds for the ``time variation" of individual objects.
Furthermore, for EK Dra, which has the most extensive dataset, we also investigate the dependence of starspot area on the rotation period to explore potential signatures of solar-like differential rotation.
The structure of this paper is as follows: Section \ref{sec:2} describes the target stars and our data analysis methods. Section \ref{sec:3} presents the results of flare detection and the correlation analysis. Section \ref{sec:4} discusses the physical implications of our findings, and we summarize our conclusions in Section \ref{sec:5}.

\section{Data and Analysis}\label{sec:2}

%\subsection{Target Stars}

\subsection{TESS Data}\label{sec:tess-data}

TESS performs high-precision photometric monitoring with four wide-field cameras equipped with the TESS filter, which covers the optical band from 6,000 to 10,000 {\AA} \citep{2015JATIS...1a4003R}. Each observing sector spans $\sim$27 days. The data used in this study were retrieved from the Mikulski Archive for Space Telescopes (MAST)\footnote{\url{https://mast.stsci.edu/portal/Mashup/Clients/Mast/Portal.html}}.

TESS observed V889 Her and DS Tuc in four sectors each, and EK Dra in twelve sectors with the 2-minute short-cadence mode. EK Dra was not included in the target list for Sector 50 and was observed only in full-frame images; to maintain uniformity, we excluded this sector from our analysis. Although some sectors provide 20-second cadence data, we did not use them because the higher cadence increases photon noise, which can negatively affect flare detection.
We primarily adopted the Presearch Data Conditioning Simple Aperture Photometry (PDC-SAP) light curves \citep{2015JATIS...1a4003R}.
However, beginning with Sector 75 in 2024, the photometric scatter in the PDC-SAP data for EK Dra appears to have increased slightly. After comparing the Simple Aperture Photometry (SAP) and PDC-SAP products, we found that the SAP data remain sufficiently reliable for flare analysis, and therefore we used SAP light curves from 2024 onward.

\subsection{Flare Detection}

We detected flares from the 2-min cadence TESS light curves using an automated pipeline, basically following the classical method employed in previous papers \citep{2012Natur.485..478M,2022ApJ...926L...5N}.
See Figures \ref{fig:lc-v889her}, \ref{fig:lc-dstuc}, and \ref{fig:lc-EKDra}
for the light curve for each stars.
The analysis code is available here\footnote{\url{https://github.com/yama662607/kyoto-flare-detection}}.
In the following, we briefly summarize the methodology used.

The light curves were processed after normalization by the mean flux. Photometric uncertainties were re-estimated from the local scatter of quiet cadences within a sliding window of $\pm 0.5$~day, and this uncertainty series was used for flare detection.
Flare candidates were identified as sequences of at least two adjacent cadences exceeding $5\sigma$. For each candidate, the event window was defined as the contiguous interval where the residual flux exceeds $1\sigma$. Each event was then re-validated using a local linear baseline estimated from pre- and post-event windows, requiring at least two cadences exceeding $3\sigma$ within the refined event.

It should be noted that DS Tuc is a binary system, in which transits can lead to false flare detections and unreliable detrended light curves \citep{2022A&A...661A.148C}. Therefore, the light curve data around the expected transit times were manually removed prior to detrending and flare detection.
In addition, because both binary components are contained within a single pixel in TESS data, flare energies and starspot areas were calculated by taking the binarity into account (Sections \ref{sec:2-3} and \ref{sec:2-4}).

\begin{figure*}
 \begin{center}
  \includegraphics[width=14cm]{figures/s0053_V889Her_lightcurve.pdf}
 \end{center}
\caption{TESS light curve of V889 Her. (Upper) The light curve normalized by averaged flux. (Lower) The detrened light curve. Red lines marks the timing of detected flares. %日本語
}\label{fig:lc-v889her}
\end{figure*}

%yamashiki
\begin{figure*}
 \begin{center}
  \includegraphics[width=14cm]{figures/s0001_DSTucA_lightcurve.pdf}
 \end{center}
\caption{The same as Figure \ref{fig:lc-v889her} but for DS Tuc. %日本語
}\label{fig:lc-dstuc}
\end{figure*}

%ishihara
\begin{figure*}
 \begin{center}
  \includegraphics[width=14cm]{figures/s0014_lightcurve.pdf}
 \end{center}
\caption{The same as Figure \ref{fig:lc-v889her} but for EK Dra. %日本語
}\label{fig:lc-EKDra}
\end{figure*}


% (日本語) フレア検出パイプラインの主要パラメータを表\ref{tab:pipeline-params}に示す。
% \begin{table}
% \caption{Key parameters of the flare detection pipeline.}
% \label{tab:pipeline-params}
% \centering
% \begin{tabular}{lccc}
% \hline
% Parameter & DS Tuc A & EK Dra & V889 Her \\
% \hline
% Flux product (late sectors) & SAP for sector $>74$ & SAP for sector $>74$ & SAP for sector $>90$ \\
% Gap threshold for offset correction (day) & 0.05 & 0.2 & 0.004 \\
% Low-pass cutoff $f_{\rm cut}$ (day$^{-1}$) & 3 & 3 & 30 \\
% Spline cutoff $f_{\rm spline}$ (day$^{-1}$) & 6 & 6 & 40 \\
% Rotation period search range (day) & 1.0--8.0 & 1.5--5.0 & 0.3--2.0 \\
% \hline
% \end{tabular}
% \end{table}

\subsection{Flare Energy}\label{sec:2-3}

We estimate the flare energy in the TESS band using the instrument response function \citep{2015JATIS...1a4003R} and a classical blackbody approximation. A fixed flare blackbody temperature of $T_{\rm flare}=10{,}000$~K is assumed \citep{2013ApJS..209....5S}. This assumption may lead to a few tens of percent error in flare energy \citep{2017ApJ...851...91N}. The time-integrated excess in the normalized light curve is converted to energy using the band-integrated luminosity ratio and the stellar radius.

In the methodology by \citet{2013ApJS..209....5S}, first the flare area is given by
\begin{eqnarray}
A_{\mathrm{flare}}(t) = C_{\mathrm{flare}}(t)\pi R_{\mathrm{star}}^2
\frac{\int R_{\lambda} B_{\lambda}(T_{\mathrm{star}}) {\rm d}\lambda}
{\int R_{\lambda} B_{\lambda}(T_{\mathrm{flare}}) {\rm d}\lambda},
\end{eqnarray}
where $C_{\mathrm{flare}}(t)$ is the relative increase in TESS-band luminosity, $R_{\mathrm{star}}$ is the stellar radius, $R_{\lambda}$ is the TESS response function, and $B_{\lambda}$ is the Planck function. The flare luminosity is then
\begin{equation}
L_{\rm flare}(t) = \sigma_{\rm SB} T_{\rm flare}^4 A_{\rm flare}(t),
\label{eq:L_flare}
\end{equation}
where $\sigma_{\rm SB}$ is the Stefan--Boltzmann constant.
Consequently, the total flare energy is obtained by integrating over the flare duration,
\begin{equation}
E_{\rm flare} = \int L_{\rm flare}(t) {\rm d}t.
\end{equation}
Under this assumption, the determination of flare energies is highly precise. Therefore, we do not assign formal error bars to the energies in this work. However, as noted above, the assumption of a 10,000~K blackbody is not necessarily valid as in M-dwarf flares \citep{2024LRSP...21....1K,2025PASJ...77.1025I} and solar flares \citep{2017ApJ...851...91N}, and systematic uncertainties at the factor-of-a-few level may be present.

\subsection{Spot Area}\label{sec:2-4}

Following the method by \citet{2017PASJ...69...41M}, we estimate the starspot area $A_\mathrm{spot}$ from the relative amplitude of rotational modulation, $\Delta F/F$, in the light curve.
\begin{align}
    A_\mathrm{spot} = \left( \dfrac{\Delta F}{F}\right) A_\mathrm{star}  \frac{T_\mathrm{star}^4}{T_\mathrm{star}^4-T_\mathrm{spot}^4}. \label{eq:A-Tspot}
\end{align}
where $A_\mathrm{star}$ is the stellar projected area.
%黒点の面積は恒星の温度から推測し、ここでは以下の式を用いる。
We derived the spot temperature $T_{\rm spot}$ from the empirical equation \citep{2005LRSP....2....8B,2017PASJ...69...41M} using the stellar effective temperature $T_\mathrm{star}$:
\begin{align}
T_\mathrm{spot} = - 3.58\times10^{-5} T_\mathrm{star}^2 +0.751 {T_\mathrm{star}} + 808, \label{eq:Tspot-Tphot}
\end{align}
The TESS light curves have very high photometric precision, and the measurement uncertainties are negligible compared to the signal amplitudes. However, additional and potentially large uncertainties arise from assumptions such as the unspotted flux level and the presence of multiple spot components (e.g., \cite{2018ApJ...865..142B,2019ApJ...871..187N}). Therefore, we do not derive error bars to these starspot area in this work, but note that systematic uncertainties may affect the overall results.


\subsection{Effective Observing Time and Flare Frequency}

We computed the effective observing time by multiplying the total number of valid data points used for flare detection by the integration time. Because 2-minute-cadence TESS light curves are used, the effective observing time is defined as the number of data points multiplied by 2 minutes.
The flare frequency is derived by dividing the number of flares, $N_{\rm flare}$, by the effective observing time. The uncertainty in the frequency is estimated assuming Poisson statistics as $\sqrt{N_{\rm flare}+1}$.

\subsection{Rotation Period}

We estimated the stellar rotation period ($P_{\rm rot}$) from the (non-detrended) normalized light curve using the Lomb--Scargle periodogram. The period corresponding to the maximum power is adopted as an rotational period. The period uncertainty is estimated by fitting a Gaussian profile to the primary peak of the Lomb--Scargle power spectrum and converting the best-fit width (1$\sigma$ in frequency) into an uncertainty in period.


\section{Result}\label{sec:3}

\subsection{Detected Flares}\label{sssec:3-1}

The top panels of Figures~\ref{fig:lc-v889her}, \ref{fig:lc-dstuc}, and \ref{fig:lc-EKDra} present example light curves of V889~Her, DS~Tuc, and EK~Dra, respectively, with time on the horizontal axis and normalized flux on the vertical axis. The light curves exhibit quasi-periodic variations caused by rotational modulation of starspots \citep{2013ApJ...771..127N,2023ApJ...948...64I}.
The bottom panels show the corresponding detrended light curves, in which these quasi-periodic variations have been removed. In these detrended data, sporadic flux enhancements are interpreted as stellar flares. The red markers indicate the timings of the flares detected by our automated procedures.
The automated procedure successfully identifies the flare events that are also evident by visual inspection in the original light curves.


Figures~\ref{fig:Cumulaive distribution of V889 Her}, \ref{fig:Cumulative distribution of DS Tuc}, and \ref{fig:Cumulative distribution of EK Dra} show the cumulative flare frequency distributions as a function of bolometric flare energy. The distributions follow power-law forms, while their absolute levels vary from sector to sector (cf. \cite{2022NatAs...6..241N}). The detected number of flares, flare frequency ($>5\times10^{33}$~erg), starspot area, and rotation period are summarized in Table~\ref{tab:flareparam}.


\begin{table}
\caption{Flare parameters.\label{tab:flareparam}}
\centering
\begin{tabular}{ccccc}
\hline
Sect. & N$_{\rm flare}$\footnotemark[1] & Freq$_{\rm flare}$ & $A_{\rm spot}$ & $P_{\rm rot}$ \\
& & [d$^{-1}$] & [$10^{21}$ cm$^{2}$] & [d] \\
\hline
\textsf{V889 Her} \\
26 & 15 & 0.64 & 1.96 & 1.37$\pm$0.03\\
40 & 12 & 0.44 & 1.16 & 1.33$\pm$0.02\\ %0.03\\
53 & 8 & 0.37 & 1.60 & 1.33$\pm$0.03\\
80 & 5 & 0.28 & 0.75 & 1.41$\pm$0.03\\
\textsf{DS Tuc} \\
1 & 7 & 0.28 & 3.86 & 2.85$\pm$0.14\\ %0.16 \\
27 & 9 & 0.39 & 5.57 & 3.63$\pm$0.20\\ %0.57\\
28 & 10 & 0.48 & 4.26 & 3.53$\pm$0.19\\ %0.23\\
67 & 5 & 0.24 & 2.75 & 3.01$\pm$0.12\\ %0.14\\
68 & 8 & 0.38 & 3.60 & 2.86$\pm$0.13\\ %0.15\\
\textsf{EK Dra} \\
14 & 5 & 0.19 & 0.86 & 2.64$\pm$0.09\\%0.11 \\
15 & 5 & 0.20 &1.20 & 2.67$\pm$0.10\\%0.12  \\
16 & 2 & 0.086 & 0.66 & 2.55$\pm$0.09\\ %0.11\\
21 & 10 & 0.39 & 1.84 & 2.64$\pm$0.09\\ %0.11\\
22 & 4 & 0.17 & 1.28 & 2.62$\pm$0.09\\ %0.11\\
23 & 4 & 0.21 & 0.74 & 2.58$\pm$0.09\\ %0.11\\
41 & 7 & 0.28 & 1.08 & 2.74$\pm$0.10\\ %0.12\\
48 & 3 & 0.14 & 0.91 & 2.71$\pm$0.10\\ %0.12\\
49 & 2 & 0.11 & 0.78 & 2.85$\pm$0.11\\ %0.13\\
75 & 5 & 0.18 & 0.93 & 2.57$\pm$0.09\\ %0.10\\
76 & 6 & 0.23 & 1.14 & 2.63$\pm$0.10\\ %0.12\\
77 & 4 & 0.22 & 1.21 & 2.62$\pm$0.07\\ %0.21\\
\hline
\end{tabular}
\begin{minipage}{0.5\textwidth}
\footnotesize
\centering
\textbf{Note.} The number of flares with energy $>5\times10^{33}$ erg.
\end{minipage}
\end{table}


%%%%%%%%%%%%%%%%%%%%%%%%%%%%%%%%%%%%%%%
\begin{figure}
 \begin{center}
  \includegraphics[width=8cm]{figures/flare_cumenergy_V889Her.pdf}
 \end{center}
\caption{Cumulative flare energy distribution of V889 Her. Each color represents a different TESS sector. The dashed line indicates the energy threshold of $5\times10^{33}$~erg used to calculate the flare frequency. %日本語
}\label{fig:Cumulaive distribution of V889 Her}
\end{figure}

\begin{figure}
 \begin{center}
  \includegraphics[width=8cm]{figures/flare_cumenergy_DSTuc.pdf}
 \end{center}
\caption{The same as Figure \ref{fig:Cumulaive distribution of V889 Her} but for DS Tuc. %日本語
}\label{fig:Cumulative distribution of DS Tuc}
\end{figure}

\begin{figure}
 \begin{center}
  \includegraphics[width=8cm]{figures/flare_cumenergy_EKDra.pdf}
 \end{center}
\caption{The same as Figure \ref{fig:Cumulaive distribution of V889 Her} but for EK Dra. %日本語
}\label{fig:Cumulative distribution of EK Dra}
\end{figure}


\subsection{Relationship between Spot Area and Flare}\label{sssec:3-2}

Figure~\ref{fig:spotarea-flarefreq} shows the relationship between the starspot area and the occurrence frequency of flares with energies larger than $5\times10^{33}$~erg, for each star and each TESS sector.
We found that, for all stars, the flare frequency distributions are affected by incompleteness below $5\times10^{33}$~erg due to limited detection sensitivity (see Figures~\ref{fig:Cumulaive distribution of V889 Her}, \ref{fig:Cumulative distribution of DS Tuc}, and \ref{fig:Cumulative distribution of EK Dra}). We therefore adopt $5\times10^{33}$~erg as a uniform energy threshold.
For all stars, a positive correlation between the starspot area and the flare frequency can be seen. We fitted the data for each star with a power-law function of the form $y = ax^{b}$.
As a result, we obtained
$a = 0.35 \pm 0.04$ and $b = 0.71 \pm 0.27$ for V889~Her,
$a = 0.14 \pm 0.06$ and $b = 0.65 \pm 0.33$ for DS~Tuc, and
$a = 0.18 \pm 0.01$ and $b = 1.11 \pm 0.15$ for EK~Dra.
%As a result, we obtained $a = 3.45 \times 10^{-1} \pm 4.49 \times 10^{-2}$ and $b = 0.71 \pm 0.28$ for V889~Her, $a = 9.10 \times 10^{-2} \pm 4.49 \times 10^{-2}$ and $b = 0.93 \pm 0.34$ for DS~Tuc, and $a = 1.92 \times 10^{-1} \pm 8.72 \times 10^{-3}$ and $b = 1.05 \pm 0.14$ for EK~Dra.
We found that the power-law index $b$ is almost unity and common among the stars, while the distribution of flare occurrence frequency at a given starspot area differs from star to star.


\begin{figure}
 \begin{center}
  \includegraphics[width=8cm]{figures/analysis_result_freq_with_error_plot.pdf}
 \end{center}
\caption{Relationship between starspot area and flare frequency ($>5\times10^{33}$ erg).
The blue, black, and red symbols correspond to V889~Her, DS~Tuc, and EK~Dra, respectively. Each point corresponds to a different TESS sector.
Error bars on the flare frequency are Poisson uncertainties, estimated as
$\sigma_{y}=\sqrt{N_{\mathrm{flare}}+1}/T_{\mathrm{obs}}$.
The data for each star are fitted with a power-law function of the form $y = a x^{b}$ as indicated with a line.
%The parameters are as follows: V889 Her: a = $3.45\times10^{-1} \pm4.49\times10^{-2}$, $b = 0.71\pm0.28$, DS Tuc: a = $9.10\times10^{-2} \pm 4.49\times10^{-2}$, b =$0.93\pm0.34$ , EK Dra: a = $1.92\times10^{-1}\pm8.72\times10^{-3}$, b = $1.05\pm0.14$
}\label{fig:spotarea-flarefreq}
\end{figure}

Figure~\ref{fig:spotarea-totalene} shows the relationship between the starspot area and the total energy of flares with energies larger than $5\times10^{33}$~erg, plotted in the same manner as in Figure~\ref{fig:spotarea-flarefreq}.
For all stars, a positive correlation is observed, and the data were fitted with the same power-law function, $y = ax^{b}$.
The best-fit parameters are
$a = 2.1 \pm 0.2$ and $b = 1.1 \pm 0.2$ for V889~Her,
$a = 0.33 \pm 0.09$ and $b = 1.4 \pm 0.2$ for DS~Tuc, and
$a = 0.72 \pm 0.02$ and $b = 1.00 \pm 0.09$ for EK~Dra.
%The best-fit parameters are $a = 2.13 \pm 0.18$  and $b = 1.12 \pm 0.16$ for V889~Her, $a = 0.30 \pm 0.08$  and $b = 1.42 \pm 0.19$ for DS~Tuc, and $a = 0.73 \pm 0.02 $ and $b = 0.99 \pm 0.09$ for EK~Dra.
While the power-law index $b$ is almost common among the stars, the distribution of the total flare energy as a function of starspot area is also found to differ among the stars.

\begin{figure}
 \begin{center}
  \includegraphics[width=8cm]{figures/analysis_result_totalene_plot.pdf}
 \end{center}
\caption{Relationship between starspot area and total flare energy ($>5\times10^{33}$). The data for each star are fitted with a power-law function of the form $y = a x^{b}$ as indicated with a line.
%$y = a x^b$, V889 Her: a = $2.13 \pm1.84\times10^{-1}$, $b = 1.12\pm0.16$, DS Tuc: a = $2.97\times10^{-1} \pm 8.42\times10^{-2}$, b =$1.42\pm0.19$ , EK Dra: a = $7.31\times10^{-1}\pm2.33\times10^{-2}$, b = $0.99\pm0.09$ %日本語
}\label{fig:spotarea-totalene}
\end{figure}


Figure~\ref{fig:spotarea-maxene} shows the relationship between the starspot area and the maximum flare energy observed in each sector, plotted in the same manner as in Figure~\ref{fig:spotarea-flarefreq}. In contrast to Figures~\ref{fig:spotarea-flarefreq} and~\ref{fig:spotarea-totalene}, no clear correlation between the starspot area and the maximum flare energy is found for any of the stars.

\begin{figure}
 \begin{center}
  \includegraphics[width=8cm]{figures/analysis_result_maxene_plot.pdf}
 \end{center}
\caption{Relationship between starspot area and maximum flare energy  %日本語
}\label{fig:spotarea-maxene}
\end{figure}


\subsection{Starspot Activity as a Function of Rotational Period: Case for EK Dra}


On the present-day Sun, starspots and solar flares occur preferentially at mid-latitudes (cf. \cite{2019LRSP...16....3T}), whereas high-latitude and near-equatorial activity is less common.
Whether a similar latitudinal preference exists on other active stars remains unclear, although some evidence is obtained for a limited stars (e.g., \cite{2017ApJ...846...99M}).
Because the Sun exhibits differential rotation, with longer rotation periods at higher latitudes, variations in measured stellar rotation periods may provide indirect information on starspot latitudes if a comparable differential rotation profile applies.

To explore this possibility, we analyzed EK~Dra using long-term photometric data. Rotation periods were derived for each observational sector and compared with the corresponding starspot areas to examine possible latitudinal trends.

Figure~\ref{fig:rotper-spotarea} shows starspot area as a function of sector-by-sector rotation period. Although the period uncertainties are large, starspot areas appear to be enhanced at intermediate rotation periods. This tentative trend suggests that, as on the Sun, larger starspots on EK~Dra may preferentially emerge at mid-latitudes.

\begin{figure}
 \begin{center}
  \includegraphics[width=8cm]{figures/period_vs_spotarea.pdf}
 \end{center}
\caption{Relationship between rotational period and starspot area for EK~Dra. Each point corresponds to a different TESS sector. %日本語
}\label{fig:rotper-spotarea}
\end{figure}


\section{Discussion}\label{sec:4}

Figures~\ref{fig:spotarea-flarefreq} and \ref{fig:spotarea-totalene} show that both the flare occurrence rate and the total flare energy exhibit positive correlations with starspot area. This behavior closely resemble the well-established relationship between solar flares and sunspots \citep{2000ApJ...540..583S,2019LRSP...16....3T}, and indicates that, on young solar-type stars as well, superflares are powered by the release of magnetic energy stored in large starspot groups.
These correlations further imply that temporal variations in starspot area (i.e., the amount of stored magnetic energy) can lead to substantial changes in the flare occurrence rate (i.e., the magnetic energy release rate). We note, however, that no clear signatures of activity-cycle modulation are detected in our sample, despite previous reports of an $\sim$8-yr activity cycle for EK~Dra \citep{2025arXiv251203830G}, which should be largely covered by the TESS temporal baseline.


Figures~\ref{fig:spotarea-flarefreq} and \ref{fig:spotarea-totalene} further reveal that the distributions differ among the three stars. In particular, DS~Tuc tends to exhibit either a larger inferred starspot area or a lower estimated flare frequency compared to EK~Dra and V889~Her (with V889~Her possibly showing a slightly lower flare frequency than EK~Dra).
We interpret these differences primarily as observational and methodological effects. First, the TESS light curve of DS~Tuc contains flux contributions from both DS~Tuc~A and DS~Tuc~B. Consequently, starspot areas derived under the assumption of a single star are likely overestimated. Second, the combined flux from the two components increases the mean system brightness, reducing the relative amplitude of individual flare brightenings compared to those in single-star systems such as EK~Dra and V889~Her. This dilution makes low-energy flares more difficult to detect and can lead to an underestimation of the flare occurrence rate.
In addition, DS~Tuc is more distant than the other two targets (Table~1), and therefore has a lower apparent brightness. As a result, photon-counting statistics and instrumental noise become more important, further decreasing the detection efficiency for low-energy flares. Taken together, these effects provide a natural explanation for the systematically lower apparent flare frequency and/or larger inferred spot area of DS~Tuc relative to the other stars.

Finally, we find no clear correlation between starspot area and the maximum flare energy. One plausible explanation is that photometric light curves alone cannot distinguish whether the observed rotational modulation is produced by a single exceptionally large spot group or by an ensemble of multiple moderately sized spots. As a result, the inferred spot area does not necessarily reflect the size of the largest magnetic structure capable of powering the most energetic flares.
Moreover, as shown in Figures~\ref{fig:Cumulaive distribution of V889 Her}--\ref{fig:Cumulative distribution of EK Dra}, the highest-energy flares are intrinsically rare, and the number of such events in our sample is limited. Consequently, statistical uncertainties are large and stochastic effects due to small-number statistics cannot be ruled out. Together, these factors likely contribute to the absence of an apparent correlation between starspot area and maximum flare energy. %Longer-term monitoring with improved statistics will be essential for placing stronger constraints on these interpretations.


\section{Summary and Conclusion}\label{sec:5}

We investigated the time-resolved relationship between starspot area and flare activity on individual young solar-type stars using $\sim$7 years of TESS photometry. Focusing on V889~Her, DS~Tuc, and EK~Dra, we automatically detected flares and derived the flare frequency, starspot area, and rotation period for each $\sim$27-day TESS sector.

We find that both the starspot area and the flare frequency vary substantially from sector to sector, although no clear activity-cycle-like modulation is detected. For all three stars, the flare occurrence rate above $5\times10^{33}$~erg exhibits a positive correlation with starspot area, and the relation is well described by a power law with an index close to unity. This result provides direct, time-resolved evidence that stellar flare activity on young solar-type stars is causally linked to the amount of magnetic energy stored in starspots, closely analogous to the solar case.
We find no clear correlation between starspot area and the maximum flare energy. This is plausibly explained by the inability of photometric light curves to isolate the largest individual spot groups, together with the intrinsically low occurrence rate of the most energetic flares and resulting small-number statistics.
For EK~Dra, we further examined the relationship between starspot area and sector-by-sector rotation period and found a tentative enhancement of spot area at intermediate rotation periods. Assuming differential rotation, this may indicate that large starspots preferentially form at mid-latitudes, similar to the Sun, although the current uncertainties prevent a definitive conclusion.


Overall, our results indicate that the fundamental connection between starspots and flares established for the Sun extends to the extreme magnetic activity of young solar-type stars. Future progress will require simultaneous spectroscopic mapping of starspots (e.g., Doppler imaging, cf. \cite{2025arXiv251203830G,2025arXiv251112190L}) and continued long-term photometric monitoring, as well as expanding the sample to a larger population of Kepler and TESS targets, in order to place stronger constraints on the spatial distribution of starspots and the long-term evolution of the spot–flare connection.
%%%%%%%%%%%%%%%%%%%%%%%%%%%%%%%%%%%%%%%


% See the instraction below for "Alt text"
% https://academic.oup.com/pasj/pages/General_Instructions#Figures%20and%20Illustrations



%%%%%%%%%%%%%%%%%%%%%%%%%%%%%%%%%%%%%%%

\begin{comment}
\section*{Supplementary data}
The following supplementary data is available at PASJ online.
E-table 1
\end{comment}

\begin{ack}
This work was supported by JSPS (Japan Society for the Promotion of Science) KAKENHI Grant Numbers 21J00316, 24H00248, 24K00680, and 25K01041 (K.N.).
This work was supported by the Operation Management Laboratory (OML) of the National Institutes of Natural Sciences (NINS), Japan (K.N.).
This paper includes data collected with the TESS mission, obtained from the MAST data archive at the Space Telescope Science Institute (STScI). Funding for the TESS mission is provided by the NASA Explorer Program. STScI is operated by the Association of Universities for Research in Astronomy, Inc., under NASA contract NAS 5-26555.
%Some of the data presented in this paper were obtained from the Mikulski Archive for Space Telescopes (MAST) at the Space Telescope Science Institute.
The authors acknowledge ideas from the participants in the workshop ``Blazing Paths to Observing Stellar and Exoplanet Particle Environments" organized by the W.M. Keck Institute for Space Studies.
The authors also would like to acknowledge the the relevant discussions in the International Space Science Institute (ISSI)
Workshop ``Stellar Magnetism and its Impact on (Exo)Planets (\url{https://workshops.issibern.ch/stellar-magnetism/})".
\end{ack}

% \bibliographystyle{****}
% \bibliography{****}
%\bibliography{tess.student.sun-like-star}{}

%bibliographystyle{pasj}
%\bibliographystyle{pasj}
%\bibliographystyle{plainnat}
%\bibliography{tess_student_sun-like-star}


\begin{thebibliography}{}
\bibitem[Airapetian et al.(2016)]{2016NatGe...9..452A} Airapetian, V.~S., Glocer, A., Gronoff, G., et al.\ 2016, Nature Geoscience, 9, 6, 452. doi:10.1038/ngeo2719

\bibitem[Airapetian et al.(2020)]{2020IJAsB..19..136A} Airapetian, V.~S., Barnes, R., Cohen, O., et al.\ 2020, International Journal of Astrobiology, 19, 2, 136. doi:10.1017/S1473550419000132

\bibitem[Ara{\'u}jo \& Valio(2021)]{2021ApJ...922L..23A} Ara{\'u}jo, A. \& Valio, A.\ 2021, \apjl, 922, 2, L23. doi:10.3847/2041-8213/ac3767

\bibitem[Atri(2017)]{2017MNRAS.465L..34A} Atri, D.\ 2017, \mnras, 465, 1, L34. doi:10.1093/mnrasl/slw199

\bibitem[Aulanier et al.(2013)]{2013A&A...549A..66A} Aulanier, G., D{\'e}moulin, P., Schrijver, C.~J., et al.\ 2013, \aap, 549, A66. doi:10.1051/0004-6361/201220406

\bibitem[Ayres(1997)]{1997JGR...102.1641A} Ayres, T.~R.\ 1997, \jgr, 102, E1, 1641. doi:10.1029/96JE03306

\bibitem[Basri(2018)]{2018ApJ...865..142B} Basri, G.\ 2018, \apj, 865, 2, 142. doi:10.3847/1538-4357/aade45

\bibitem[Benomar et al.(2018)]{2018Sci...361.1231B} Benomar, O., Bazot, M., Nielsen, M.~B., et al.\ 2018, Science, 361, 6408, 1231. doi:10.1126/science.aao6571

\bibitem[Bell et al.(2015)]{2015MNRAS.454..593B} Bell, C.~P.~M., Mamajek, E.~E., \& Naylor, T.\ 2015, \mnras, 454, 1, 593. doi:10.1093/mnras/stv1981

\bibitem[Benz \& G{\"u}del(2010)]{2010ARA&A..48..241B} Benz, A.~O. \& G{\"u}del, M.\ 2010, \araa, 48, 241. doi:10.1146/annurev-astro-082708-101757

\bibitem[Berdyugina(2005)]{2005LRSP....2....8B} Berdyugina, S.~V.\ 2005, Living Reviews in Solar Physics, 2, 1, 8. doi:10.12942/lrsp-2005-8

\bibitem[Cliver et al.(2022)]{2022LRSP...19....2C} Cliver, E.~W., Schrijver, C.~J., Shibata, K., et al.\ 2022, Living Reviews in Solar Physics, 19, 1, 2. doi:10.1007/s41116-022-00033-8

\bibitem[Colombo et al.(2022)]{2022A&A...661A.148C} Colombo, S., Petralia, A., \& Micela, G.\ 2022, \aap, 661, A148. doi:10.1051/0004-6361/202243086

\bibitem[Davenport(2016)]{2016ApJ...829...23D} Davenport, J.~R.~A.\ 2016, \apj, 829, 1, 23. doi:10.3847/0004-637X/829/1/23

\bibitem[Emslie et al.(2012)]{2012ApJ...759...71E} Emslie, A.~G., Dennis, B.~R., Shih, A.~Y., et al.\ 2012, \apj, 759, 1, 71. doi:10.1088/0004-637X/759/1/71

\bibitem[Gaia Collaboration et al.(2021)]{2021A&A...649A...1G} Gaia Collaboration, Brown, A.~G.~A., Vallenari, A., et al.\ 2021, \aap, 649, A1. doi:10.1051/0004-6361/202039657

\bibitem[G{\"o}rgei et al.(2025)]{2025arXiv251203830G} G{\"o}rgei, A., Kriskovics, L., Vida, K., et al.\ 2025, , arXiv:2512.03830. doi:10.48550/arXiv.2512.03830

\bibitem[G{\"u}del(2007)]{2007LRSP....4....3G} G{\"u}del, M.\ 2007, Living Reviews in Solar Physics, 4, 1, 3. doi:10.12942/lrsp-2007-3

\bibitem[H{\o}g et al.(2000)]{2000A&A...355L..27H} H{\o}g, E., Fabricius, C., Makarov, V.~V., et al.\ 2000, \aap, 355, L27.

\bibitem[Ichihara et al.(2025)]{2025PASJ...77.1025I} Ichihara, S., Nogami, D., Namekata, K., et al.\ 2025, \pasj, 77, 5, 1025. doi:10.1093/pasj/psaf080

\bibitem[Ikuta et al.(2023)]{2023ApJ...948...64I} Ikuta, K., Namekata, K., Notsu, Y., et al.\ 2023, \apj, 948, 1, 64. doi:10.3847/1538-4357/acbd36

\bibitem[J{\"a}rvinen et al.(2018)]{2018A&A...620A.162J} J{\"a}rvinen, S.~P., Strassmeier, K.~G., Carroll, T.~A., et al.\ 2018, \aap, 620, A162. doi:10.1051/0004-6361/201833496

\bibitem[K{\"o}nig et al.(2005)]{2005A&A...435..215K} K{\"o}nig, B., Guenther, E.~W., Woitas, J., et al.\ 2005, \aap, 435, 1, 215. doi:10.1051/0004-6361:20040462

\bibitem[Kowalski(2024)]{2024LRSP...21....1K} Kowalski, A.~F.\ 2024, Living Reviews in Solar Physics, 21, 1, 1. doi:10.1007/s41116-024-00039-4

\bibitem[Lee et al.(2025)]{2025arXiv251112190L} Lee, S., Bahar, E., {\c{S}}enavc{\i}, H.~V., et al.\ 2025, , arXiv:2511.12190. doi:10.48550/arXiv.2511.12190

\bibitem[Lindegren et al.(2018)]{2018A&A...616A...2L} Lindegren, L., Hern{\'a}ndez, J., Bombrun, A., et al.\ 2018, \aap, 616, A2. doi:10.1051/0004-6361/201832727

\bibitem[Loyd et al.(2025)]{2025kiss.rept.....L} Loyd, R.~O.~P., Shkolnik, E.~L., Lazio, J., et al.\ 2025, Report prepared for the W. M. Keck Institute for Space Studies (KISS), California Institute of Technology, by R.O.P. Loyd et al, 2025.. doi:10.26206/gmhk5-amp17

\bibitem[Maehara et al.(2012)]{2012Natur.485..478M} Maehara, H., Shibayama, T., Notsu, S., et al.\ 2012, \nat, 485, 7399, 478. doi:10.1038/nature11063

\bibitem[Maehara et al.(2017)]{2017PASJ...69...41M} Maehara, H., Notsu, Y., Notsu, S., et al.\ 2017, \pasj, 69, 3, 41. doi:10.1093/pasj/psx013

\bibitem[Morris et al.(2017)]{2017ApJ...846...99M} Morris, B.~M., Hebb, L., Davenport, J.~R.~A., et al.\ 2017, \apj, 846, 2, 99. doi:10.3847/1538-4357/aa8555

\bibitem[Namekata et al.(2017)]{2017ApJ...851...91N} Namekata, K., Sakaue, T., Watanabe, K., et al.\ 2017, \apj, 851, 2, 91. doi:10.3847/1538-4357/aa9b34

\bibitem[Namekata et al.(2019)]{2019ApJ...871..187N} Namekata, K., Maehara, H., Notsu, Y., et al.\ 2019, \apj, 871, 2, 187. doi:10.3847/1538-4357/aaf471

\bibitem[Namekata et al.(2021)]{2022NatAs...6..241N} Namekata, K., Maehara, H., Honda, S., et al.\ 2021, Nature Astronomy, 6, 241. doi:10.1038/s41550-021-01532-8

\bibitem[Namekata et al.(2022)]{2022ApJ...926L...5N} Namekata, K., Maehara, H., Honda, S., et al.\ 2022, \apjl, 926, 1, L5. doi:10.3847/2041-8213/ac4df0

\bibitem[Namekata et al.(2025)]{2025ApJ...993...80N} Namekata, K., Maehara, H., Notsu, Y., et al.\ 2025, \apj, 993, 1, 80. doi:10.3847/1538-4357/adfe70

\bibitem[Newton et al.(2019)]{2019ApJ...880L..17N} Newton, E.~R., Mann, A.~W., Tofflemire, B.~M., et al.\ 2019, \apjl, 880, 1, L17. doi:10.3847/2041-8213/ab2988

\bibitem[Notsu et al.(2013)]{2013ApJ...771..127N} Notsu, Y., Shibayama, T., Maehara, H., et al.\ 2013, \apj, 771, 2, 127. doi:10.1088/0004-637X/771/2/127

\bibitem[Notsu et al.(2015)]{2015PASJ...67...33N} Notsu, Y., Honda, S., Maehara, H., et al.\ 2015, \pasj, 67, 3, 33. doi:10.1093/pasj/psv002

\bibitem[Notsu et al.(2019)]{2019ApJ...876...58N} Notsu, Y., Maehara, H., Honda, S., et al.\ 2019, \apj, 876, 1, 58. doi:10.3847/1538-4357/ab14e6

\bibitem[Okamoto et al.(2021)]{2021ApJ...906...72O} Okamoto, S., Notsu, Y., Maehara, H., et al.\ 2021, \apj, 906, 2, 72. doi:10.3847/1538-4357/abc8f5

\bibitem[Ricker et al.(2015)]{2015JATIS...1a4003R} Ricker, G.~R., Winn, J.~N., Vanderspek, R., et al.\ 2015, Journal of Astronomical Telescopes, Instruments, and Systems, 1, 014003. doi:10.1117/1.JATIS.1.1.014003

\bibitem[Sammis et al.(2000)]{2000ApJ...540..583S} Sammis, I., Tang, F., \& Zirin, H.\ 2000, \apj, 540, 1, 583. doi:10.1086/309303

\bibitem[Schaefer et al.(2000)]{2000ApJ...529.1026S} Schaefer, B.~E., King, J.~R., \& Deliyannis, C.~P.\ 2000, \apj, 529, 2, 1026. doi:10.1086/308325

\bibitem[Segura et al.(2010)]{2010AsBio..10..751S} Segura, A., Walkowicz, L.~M., Meadows, V., et al.\ 2010, Astrobiology, 10, 7, 751. doi:10.1089/ast.2009.0376

\bibitem[Strassmeier et al.(2003)]{2003A&A...411..595S} Strassmeier, K.~G., Pichler, T., Weber, M., et al.\ 2003, \aap, 411, 595. doi:10.1051/0004-6361:20031538

\bibitem[Shibata \& Magara(2011)]{2011LRSP....8....6S} Shibata, K. \& Magara, T.\ 2011, Living Reviews in Solar Physics, 8, 1, 6. doi:10.12942/lrsp-2011-6

\bibitem[Shibata et al.(2013)]{2013PASJ...65...49S} Shibata, K., Isobe, H., Hillier, A., et al.\ 2013, \pasj, 65, 3, 49. doi:10.1093/pasj/65.3.49

\bibitem[Shibayama et al.(2013)]{2013ApJS..209....5S} Shibayama, T., Maehara, H., Notsu, S., et al.\ 2013, \apjs, 209, 1, 5. doi:10.1088/0067-0049/209/1/5

\bibitem[Tokuno et al.(2025)]{2025ApJ...985..158T} Tokuno, T., Namekata, K., Maehara, H., et al.\ 2025, \apj, 985, 2, 158. doi:10.3847/1538-4357/adce7c

\bibitem[Toriumi \& Wang(2019)]{2019LRSP...16....3T} Toriumi, S. \& Wang, H.\ 2019, Living Reviews in Solar Physics, 16, 1, 3. doi:10.1007/s41116-019-0019-7

\bibitem[Vasilyev et al.(2024)]{2024Sci...386.1301V} Vasilyev, V., Reinhold, T., Shapiro, A.~I., et al.\ 2024, Science, 386, 6727, 1301. doi:10.1126/science.adl5441

\bibitem[Waite et al.(2017)]{2017MNRAS.465.2076W} Waite, I.~A., Marsden, S.~C., Carter, B.~D., et al.\ 2017, \mnras, 465, 2, 2076. doi:10.1093/mnras/stw2731

\bibitem[Yamashita et al.(2022)]{2022PASJ...74.1295Y} Yamashita, M., Itoh, Y., \& Oasa, Y.\ 2022, \pasj, 74, 6, 1295. doi:10.1093/pasj/psac069


\end{thebibliography}

\end{document}

=== Frequently used abbreviation of journal names ===
\aj         AJ
\araa       ARA\&A
\apj        ApJ
\apjl       ApJL
\apjs       ApJS
\apss       Ap\&SS
\aap        A\&A
\aapr       A\&AR
\aaps       A\&AS
\baas       BAAS
\icarus     ICARUS
\mnras      MNRAS
\prd        Phys.\ Rev.\ D
\prl        Phys.\ Rev.\ Lett.
\pasp       PASP
\pasj       PASJ
\solphys    Sol.\ Phys.
\ssr        Space\ Sci.\ Rev.
\nat        Nature
\iaucirc    IAU\ Circ.
\gca        Geochim.\ Cosmochim.\ Acta
\jgr        J.\ Geophys.\ Res.
\nphysa     Nucl.\ Phys.\ A
\procspie   Proc.\ SPIE
\aip        AIP Conf.\ Proc.
\asp        ASP Conf.\ Ser.
=====================================================
