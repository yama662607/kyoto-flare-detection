%%%%%%%%%%%%%%%%%%%%%%%%%%%%%%%%%%%%%%%%%%%%%%%%%%%%%%%%%%%%%%%%%%%%%%%%%
%%% PASJ LaTeX template for draft(body) <2024/07/01>
%%%
%%% IMPORTANT NOTICE FOR AUTHORS
%%%  1. Do NOT use \def/\renewcommand.
%%%  2. Do NOT redefine commands provided by PASJ02.cls.
%%%  3. LETTER article must NOT exceed ``six pages'' in length in PASJ's publication layout format.
%%%    Do NOT change the default font setting of pasj02.cls to avoid obtaining an inaccurate page estimation.
%%%  4. ``\draft'' creates single column and double spaces format.
%%%
%%% Instructions to authors: https://academic.oup.com/pasj/pages/General_Instructions
%%% Author's guide (in Japanese): https://www.asj.or.jp/pasj/guide/
%%%%%%%%%%%%%%%%%%%%%%%%%%%%%%%%%%%%%%%%%%%%%%%%%%%%%%%%%%%%%%%%%%%%%%%%%
\documentclass[]{pasj02}
%\draft
\usepackage[switch,mathlines]{lineno} % add line number to manuscript
\usepackage{comment}
\usepackage{url}
%\usepackage{natbib}


\jyear{2026}
\Received{}%{yyyy/mm/dd}
\Accepted{}%{yyyy/mm/dd}
%\Published{yyyy/mm/dd}

%\graphicspath{{./}{figures/}}

% AAS .bst が出す \href, \nolinkurl 用
%\usepackage{url}
%\usepackage[hidelinks]{hyperref}

% AAS .bst は natbib 前提
%\usepackage{natbib}

% PASJ が定義した \dodoi を一旦解放して、
% AAS .bst が .bbl 冒頭で \providecommand し直せるようにする
%\makeatletter
%\let\pasj@front@dodoi\dodoi  % PASJの\dodoiを退避(必要なら後で使える)
%\let\dodoi\relax             % いったん未定義化
%\makeatother

\begin{document}

\title{Time-Resolved Connection between Starspots and Flares in Nearby Young Sun-like Stars Observed by TESS}

%%% begin:list of authors
% Do NOT capitalize all letters in "textsc".
\author{
 Daijiro \textsc{Hitotsuyanagi},\altaffilmark{1}\altemailmark \email{hitotsuyanagi.daijiro.23f@st.kyoto-u.ac.jp}
 Hiroto \textsc{Yamada},\altaffilmark{1}\altemailmark \email{yamada.hiroto.74h@st.kyoto-u.ac.jp}
 Daisuke \textsc{Yamashiki},\altaffilmark{1}\altemailmark \email{yamashiki.daisuke.24f@st.kyoto-u.ac.jp}
 Kazuma \textsc{Ishihara}\altaffilmark{1}\altemailmark \email{k.ishihara@kusastro.kyoto-u.ac.jp}
 Yosuke \textsc{Yamashiki},\altaffilmark{2,3}\altemailmark \email{yamashiki.yosuke.3u@kyoto-u.ac.jp}
 and
 Kosuke \textsc{Namekata}\altaffilmark{4,5,6,7}\altemailmark\orcid{0000-0002-1297-9485} \email{namekata@kusastro.kyoto-uac.jp}
}
\altaffiltext{1}{Department of Physics, Faculty of Science, Kyoto University, Kitashirakawa-Oiwake-cho, Sakyo-ku, Kyoto 606-8502, Japan}
\altaffiltext{2}{Graduate School of Advanced Integrated Studies in Human Survivability (GSAIS), Kyoto University, Yoshida-Nakaadachi-cho, Sakyo-ku, Kyoto 606-8306, Japan}
\altaffiltext{3}{Unit of Synergetic Studies for Space, Kyoto University, Yoshida-Honmachi, Sakyo-ku, Kyoto 606-8501, Japan}
\altaffiltext{4}{Heliophysics Science Division, NASA Goddard Space Flight Center, 8800 Greenbelt Road, Greenbelt, MD 20771, USA}
\altaffiltext{5}{The Catholic University of America, 620 Michigan Avenue, N.E. Washington, DC 20064, USA}
\altaffiltext{6}{The Hakubi Center for Advanced Research, Kyoto University, Yoshida-Honmachi, Sakyo-ku, Kyoto 606-8501, Japan}
\altaffiltext{7}{Department of Physics, Kyoto University, Kitashirakawa-Oiwake-cho, Sakyo-ku, Kyoto, 606-8502, Japan}


%\footnotetext[$\dag$]{Present address: ....}

%%% end:list of authors

%% !!! Select 3 to 5 words from PASJ's key words !!!
%% List of Key Words: https://academic.oup.com/pasj/pages/Pasj_Keywords
%% "\KeyWords{ }" always has to be placed before ``\maketitle''
\KeyWords{stars: activity --- stars: flare --- stars: late-type --- stars: solar-type --- starspots --- Sun: flares --- sunspots}

\maketitle

\begin{abstract}
Please read ``IMPORTANT NOTICE'' carefully before preparing a manuscript.

\citet{Airapetian_2016} suggests ...., ... is suggested by \citet{Airapetian_2016}.

Flare is caused by magnetic reconnection \citep{Airapetian_2016}.

Flare is caused by magnetic reconnection \citep{Airapetian_2016,Namekata_2022a}.

Flare is caused by magnetic reconnection (\cite{Airapetian_2016}; Namekata et al. 2022a)


\end{abstract}

\pagewiselinenumbers

\section{Introduction}

\noindent IMPORTANT NOTICE\\
1. Manuscript for submission must be in the same format as a published papers. \\
2. Line numbers should be added to the manuscript. \\
3. Do NOT use ``\verb|\def|, \verb|\renewcommand|''.\\
4. Do NOT redefine commands provided by pasj02.cls.


\section{Data and Analysis}\label{sec:2}

\subsection{Target Stars}

\subsection{TESS Data}

TESS performs high-precision photometric monitoring with four wide-field cameras equipped with the TESS filter, which covers the optical band from 6000 to 10 000 {\AA} \citep{2015JATIS...1a4003R}. Each observing sector spans approximately 27 days. The data used in this study were retrieved from the Mikulski Archive for Space Telescopes (MAST)\footnote{\url{https://mast.stsci.edu/portal/Mashup/Clients/Mast/Portal.html}}.

TESS observed V889 Her and DS Tuc in four sectors each, and EK Dra in twelve sectors with the 2-minute short-cadence mode. EK Dra was not included in the target list for Sector 50 and was observed only in full-frame images; to maintain uniformity, we excluded this sector from our analysis. Although some sectors provide 20-second cadence data, we did not use them because the higher cadence increases photon noise, which can negatively affect flare detection.

We primarily adopted the Presearch Data Conditioning Simple Aperture Photometry (PDC-SAP) light curves. However, beginning with Sector XX in 2024, the photometric scatter in the PDC-SAP data for EK Dra appears to have increased slightly. After comparing the Simple Aperture Photometry (SAP) and PDC-SAP products, we found that the SAP data remain sufficiently reliable for flare analysis, and therefore we used SAP light curves from 2024 onward.

\subsection{Flare Detection}

\subsection{Flare Energy}

\begin{eqnarray}
 L'_{\mathrm{star}} = \int R_{\lambda}B_{\lambda(T_{\mathrm{eff}})}d\lambda \cdot \pi R_{\mathrm{star}}^2,
 \label{L'_star}
\end{eqnarray}

\begin{eqnarray}
 L'_{\mathrm{flare}}(t) = \int R_{\lambda}B_{\lambda(T_{\mathrm{BB}}(t))}d\lambda \cdot A_{\mathrm{flare}}(t), \ \mathrm{and}
 \label{L'_star}
\end{eqnarray}

\begin{eqnarray}
 C'_{\mathrm{flare}}(t) = \frac{L'_{\mathrm{flare}}(t)}{L'_{\mathrm{star}}},
 \label{Planck function}
\end{eqnarray}

\begin{eqnarray}
 A_{\mathrm{flare}}(t) = C'_{\mathrm{flare}}(t)\pi R_{\mathrm{star}}^2\frac{\int R_{\lambda}B_{\lambda(T_{\mathrm{eff}})}d\lambda}{\int R_{\lambda}B_{\lambda(T_{\mathrm{BB}}(t))}d\lambda}.
\end{eqnarray}

\begin{equation}
L_{\rm flare} = \sigma_{\rm SB} T^4_{\rm flare} A_{\rm flare}
\label{eq:L_flare}
\end{equation}

\begin{equation}
E_{\rm flare} = \int_{\rm flare} L_{\rm flare}(t) dt
\end{equation}

\subsection{Spot Area}

\begin{align}
    A_\mathrm{spot} = \left( \dfrac{\Delta F}{F}\right) A_\mathrm{star}  \frac{T_\mathrm{phot}^4}{T_\mathrm{phot}^4-T_\mathrm{spot}^4}. \label{eq:A-Tspot}
\end{align}

\begin{align}
T_\mathrm{spot} = - 3.58\times10^{-5} T_\mathrm{star}^2 +0.751 {T_\mathrm{star}} + 808, \label{eq:Tspot-Tphot}
\end{align}



\subsection{}

\section{Result}\label{sec:3}

\subsection{Detected Flares}\label{sssec:3-1}

\begin{comment}

% Figure 1: red coloer is flares. looks almost all flares by eye was detected automatically.
Figure 1、2、3の上図は、それぞれV889 Her、DS Tuc、EK Draの光度曲線を表している。横軸が時刻、縦軸が正規化したFluxである。TESSの観測データには時々データが途切れている箇所が存在しているが、その箇所は直線で繋げている。また、自転による黒点の見え隠れにより、光度曲線に準周期的な変動が読み取れる。目で見てわかるように、ところどころFluxが跳ね上がっており、その時刻にフレアが発生していると考えられる。Figure 1、2、3の下図は、上図の光度曲線から準周期的な変動を除いたものである。下図中の赤いマークがその時刻でフレアを検出したことを表し、上図で目で確認できていたフレアが自動的に正確に検出されていることがわかる。

% Figure 4, 5, 6: flare occurrence frequency as a function of energy. energy increase, frequency decrease.
Figure 4、5、6は、それぞれV889 Her、DS Tuc、EK DraのCumulative Flare Energy Distributionを表している。横軸がフレアのエネルギー、縦軸がCumulative numberである。各図において、それぞれのセクターにおいてのCumulative Flare Energy Distributionがplotされている。それぞれのエネルギー以上の

\end{comment}


Looks each sector show different $E_{flare}$ and frequency

saturation.

\subsection{Relationship between Spot Area and Flare}\label{sssec:3-2}

Figure 7, 8, 9:
%Fig. 7 は、星ごと、セクターごとに、横軸に黒点面積、縦軸にエネルギーが 5×10^33erg 以上のフレアの発生頻度をとった図である。青色の点は V889 Her、黒色の点は DS Tuc、赤色の点は EK Dra をそれぞれ示している。いずれの星においても黒点面積とフレア発生頻度の間に正の相関が見られることが分かる。そこで、各星ごとにy=ax^bの形でフィッティングを行った。その結果、V889 Her については a=3.42×10^(-1)±2.58×10^(-2)、b=0.76±0.15、DS Tuc については a=1.05×10^(-1)±3.13×10^(-2)、b=0.84±0.20、EK Dra については a=1.96×10^(-1)±2.96×10^(-3)、b=0.98±0.04 が得られた。べきは大体同じだったが、星ごとに分布が異なる。また、Fig. 7 からは、同じ黒点面積に対するフレア発生頻度の分布が星ごとに異なっていることも示唆される。

%Fig.8は

\subsection{Flare Frequency as a Function of Spot's Rotational Period: Case for EK Dra}

motivation: we expect solar-like spot distribution on stars: mid-latitude is flare/spot active.

rotation period is related to latitude due to differential rotation -- latitude v.s spot area $\sim$ rotation period v.s. spot area. Maybe.

Figure 10:

error large, yokuwakaranai.

possible: chuuidotai de spot large?


%現在、太陽では中緯度帯で黒点が多く発生し、高緯度帯や赤道付近の低緯度帯では黒点が大きくなりにくいことが分かっている。そのため、フレアも中緯度帯で発生することが多い。これがEK Draでも同じことが言えるのかが知りたい。太陽では差動回転が観測されていて、緯度が高いほど周期が大きいことが分かっている。このことをEK Draでも活用する。セクターごとのローテーションピリオドを求めEK Draの自転周期の2.6日と比較することで、EK Draではどの緯度帯で黒点が多く発生したのかを調べた。
%図10は横軸にセクターごとのローテーションピリオド、縦軸に黒点の面積をプロットしたものである。ローテーションピリオドのエラーバーが大きいため確定的なことは言えないが、自転周期と比較することでEK Draでも太陽と同様に中緯度帯で黒点が大きいことが言えるかもしれない。
%今後、解析の精度があがることでより明確に分かるかもしれない。


\section{Discussion}\label{sec:4}

\section{Summary and Conclusion}

%%%%%%%%%%%%%%%%%%%%%%%%%%%%%%%%%%%%%%%

%Here!!!
%yamada
\begin{figure*}
 \begin{center}
  \includegraphics[width=14cm]{figures/s0053_V889Her_lightcurve.pdf}
 \end{center}
\caption{TESS light curve of V889 Her. (Upper) The light curve normalized by TESS. (Lower) The detrened light curve. Red lines marks the timing of detected flares. %日本語
}\label{fig:lc-v889her}
\end{figure*}

%yamashiki
\begin{figure*}
 \begin{center}
  \includegraphics[width=14cm]{figures/s0001_DSTucA_lightcurve.pdf}
 \end{center}
\caption{The same as Figure \ref{fig:lc-v889her} but for DS Tuc. %日本語
}\label{fig:lc-dstuc}
\end{figure*}

%ishihara
\begin{figure*}
 \begin{center}
  \includegraphics[width=14cm]{figures/s0014_lightcurve.pdf}
 \end{center}
\caption{The same as Figure \ref{fig:lc-v889her} but for EK Dra. %日本語
}\label{fig:lc-EKDra}
\end{figure*}


%%%%%%%%%%%%%%%%%%%%%%%%%%%%%%%%%%%%%%%
\begin{figure}
 \begin{center}
  \includegraphics[width=8cm]{figures/flare_cumenergy_V889Her.pdf}
 \end{center}
\caption{Cumulative Flare Energy Distribution of V889 Her. %日本語
}\label{fig:spotarea-flarefreq}
\end{figure}

\begin{figure}
 \begin{center}
  \includegraphics[width=8cm]{figures/flare_cumenergy_DSTuc.pdf}
 \end{center}
\caption{Cumulative Flare Energy Distribution of DS Tuc. %日本語
}\label{fig:spotarea-flarefreq}
\end{figure}

\begin{figure}
 \begin{center}
  \includegraphics[width=8cm]{figures/flare_cumenergy_EKDra.pdf}
 \end{center}
\caption{Cumulative Flare Energy Distribution of EK Dra. %日本語
}\label{fig:spotarea-flarefreq}
\end{figure}

\begin{figure}
 \begin{center}
  \includegraphics[width=8cm]{figures/analysis_result_freq_plot.pdf}
 \end{center}
\caption{Relationship between starspot area and flare frequency ($>5\times10^{33}$ erg). $y = a x^b$, V889 Her: a = $3.42\times10^{-1} \pm2.58\times10^{-2}$, $b = 0.76\pm0.15$, DS Tuc: a = $1.05\times10^{-1} \pm 3.13\times10^{-2}$, b =$0.84\pm0.20$ , EK Dra: a = $1.96\times10^{-1}\pm2.96\times10^{-3}$, b = $0.98\pm0.04$ %日本語
}\label{fig:spotarea-flarefreq}
\end{figure}

\begin{figure}
 \begin{center}
  \includegraphics[width=8cm]{figures/analysis_result_maxene_plot.pdf}
 \end{center}
\caption{Relationship between starspot area and max flare energy  %日本語
}\label{fig:}
\end{figure}

\begin{figure}
 \begin{center}
  \includegraphics[width=8cm]{figures/analysis_result_totalene_plot.pdf}
 \end{center}
\caption{Relationship between starspot area and total flare energy ($>5\times10^{33}$).$y = a x^b$, V889 Her: a = $2.13 \pm1.84\times10^{-1}$, $b = 1.12\pm0.16$, DS Tuc: a = $2.97\times10^{-1} \pm 8.42\times10^{-2}$, b =$1.42\pm0.19$ , EK Dra: a = $7.31\times10^{-1}\pm2.33\times10^{-2}$, b = $0.99\pm0.09$ %日本語
}\label{fig:}
\end{figure}
% See the instraction below for "Alt text"
% https://academic.oup.com/pasj/pages/General_Instructions#Figures%20and%20Illustrations

\begin{figure}
 \begin{center}
  \includegraphics[width=8cm]{figures/period_vs_spotarea.pdf}
 \end{center}
\caption{Relationship between starspot area and flare frequency ($>5\times10^{33}$ erg). %日本語
}\label{fig:spotarea-flarefreq}
\end{figure}


\begin{longtable}{lccc}
  \caption{Stellar parameters.}\label{tab:targets} \\
  \hline\noalign{\vskip3pt}
  Parameters & V889 Her (TIC 471000657) & DS Tuc (TIC 410214986) & EK Dra (TIC 159613900) \\ [2pt]
  \hline\noalign{\vskip3pt}
\endfirsthead

  \hline\noalign{\vskip3pt}
  Parameters & V889 Her (TIC 471000657) & DS Tuc (TIC 410214986) & EK Dra (TIC 159613900) \\ [2pt]
  \hline\noalign{\vskip3pt}
\endhead

  \hline\noalign{\vskip3pt}
\endfoot

  \hline\noalign{\vskip3pt}
  \multicolumn{2}{@{}l@{}}{\hbox to0pt{\parbox{160mm}{\footnotesize
  \hangindent6pt\noindent
  \hbox to6pt{\hss}\unskip%
  $^{(V1)}$Table 2 of \cite{2003A&A...411..595S};
  $^{(2)}$\cite{2020NatAs...4..650B};
  $^{(E1)}$\cite{2017MNRAS.465.2076W};
  $^{(4)}$\cite{Hog2000A&A...355L..27H};
  $^{(5)}$\cite{2000A&A...355L..27H};
  $^{(6)}$\cite{2018A&A...620A.162J};
  $^{(7)}$\cite{2005A&A...435..215K};
  $^{(V2)}$Gaia EDR3 \citep{2021A&A...649A...1G};
  $^{(9)}$\cite{2018A&A...616A...2L}.\\
  $^{\S}$Reported ages for EK Dra range from 30–125 Myr depending on the study.\\
  $^{\dag}$For flare‐energy calculations we adopt $T_{\text{eff}}\!\approx\!5700$ K from \cite{2005A&A...435..215K} for consistency with previous work.\\
  $^{\rm (D1)}$ \citet{2019ApJ...880L..17N}
  $^{\rm (D2)}$ \citet{2015MNRAS.454..593B}
  }\hss}}
\endlastfoot

Spectral Type              & G0V$^{({\rm V1})}$              &
% G6V+K3V$^{(2)}$
\textcolor{red}{G6V$\pm 1$/K3V$\pm 1$}$^{\rm (D1)}$ % wikiの参考文献ではなく、thyme参照値
& G1.5V$^{\rm(E1)}$ \\
$V_{\rm mag}$              & $7.45\pm0.04^{\rm(V1)}$      &
% $8.23\pm0.04$
\textcolor{red}{$8.55\pm0.01/9.65\pm0.03$}$^{\rm (D1)}$ %そもそもwikiになく、thysmのV_Tを参照
& $7.60\pm0.01^{(5)}$ \\
Age (Myr)                  & 30$^{\rm(V1)}$               & \textcolor{red}{$45\pm4$}$^{\rm (D2)}$%45
& 50--125$^{\rm(E1)}$\footnotemark[$\S$] \\
$T_{\text{eff}}$ (K)       & $5830\pm50^{\rm(V1)}$        &
% $5550\pm100$
\textcolor{red}{$5430\pm80/4700\pm90$}$^{\rm (D1)}$
& 5560--5750$^{(3,6,7)}$\footnotemark[$\dag$] \\
Radius ($R_{\odot}$)       & $1.09\pm0.05^{\rm(V1)}$      & \textcolor{red}{$0.96\pm0.03$/$0.86\pm0.04$} $^{\rm (D1)}$     & $0.94\pm0.07^{(3)}$ \\
Mass ($M_{\odot}$)         & $1.06\pm0.02^{\rm(V1)}$      &
% $0.96\pm0.02$
\textcolor{red}{$1.01\pm0.06/0.84\pm0.06$}$^{\rm (D1)}$
& $0.95\pm0.04^{(3)}$ \\
Distance (pc)              & $35.36\pm0.02^{\rm(V2)}$     & $44.78\pm0.03$
%wikiに参考文献なし
& $34.40\pm0.03^{(8)}$ \\
$P_{\text{rot}}$ (d)       & $1.3371\pm0.0002^{\rm(V1)}$  &
% $2.85\pm0.02$
\textcolor{red}{$2.85^{+0.04}_{-0.05}$/-}$^{\rm (D1)}$%thysm
& $2.766\pm0.002^{(3)}$ \\
$v\sin i$ (km s$^{-1}$)    & $39.0\pm0.5^{\rm(V1)}$       &
% $15.9\pm0.2$
\textcolor{red}{$17.8\pm0.2/14.4\pm0.3$}$^{\rm (D1)}$%thysm
& $16.4\pm0.1^{(3)}$ \\
%RV (km s$^{-1}$)           & $-23.6\pm1.5^{(1)}$      & $^$        & $-20.687\pm0.004^{(9)}$ \\
Inclination (deg)          & $\approx55^{\rm(V1)}$        &
% $-$
\textcolor{red}{$>82^{\circ}/-$}$^{\rm (D1)}$
& $60\pm5^{(3)}$ \\
Binarity                   & single$^{\rm(V1)}$           & binary/exoplanet           & low-mass companion$^{(3)}$ \\
\end{longtable}



\begin{table*}
  \tbl{First tabular.\footnotemark[$*$] }{%
  \begin{tabular}{cccccc}
      \hline
      Name & Sect. & N$_{\rm flare}$ & Freq$_{\rm flare}$ & $A_{\rm spot}$ & $P_{\rm rot}$  \\
       &  & ($>5\times10^{33}$ erg) & [d$^{-1}$] & [$10^{21}$ cm$^{2}$] & [d]  \\
      \hline
      V889 Her & 26 & 29 & ddd & & \\
       & 40 & 18 & hhh & &  \\
       & 53 & 17 & \\
       & 80 & 10 & \\
      DS Tuc & 1 & 10 & ddd & & \\
       & 27 & 13 & hhh & &  \\
       & 28 & 12 &\\
       & 67 & 8 &\\
       & 68 & 11 &....\\
      EK Dra & 14 & 18 & ddd & & \\
       & 15 & 19 & hhh & &  \\
       & 16 & 11 &\\
       & 21 & 22 &\\
       & 22 & 11 &\\
       & 23 & 10 &\\
       & 41 & 15 &\\
       & 48 & 13 &\\
       & 49 & 11 &\\
       & 75 & 10 &\\
       & 76 & 15 &\\
       & 77 & 9  &\\
      \hline
    \end{tabular}}\label{tab:flareparam}
\begin{tabnote}
\footnotemark[$*$] Brief explanation of this table.  \\
\footnotemark[$\dag$] Explanation of value 3.
%\footnotemark[$\ddag$]  ... \\
%\footnotemark[$\S$]  ... \\
%\footnotemark[$\|$]  ... \\
%\footnotemark[$\sharp$]  ... \\
%\footnotemark[$**$]  ... \\
%\footnotemark[$\dag\dag$]  ... \\
\end{tabnote}
\end{table*}



%%%%%%%%%%%%%%%%%%%%%%%%%%%%%%%%%%%%%%%

\begin{comment}
\section*{Supplementary data}
The following supplementary data is available at PASJ online.
E-table 1
\end{comment}

\begin{ack}
This work was supported by JSPS (Japan Society for the Promotion of Science) KAKENHI Grant Numbers 21J00316, 25K01041, 24H00248, and 24K00680 (K.N.).
This work was supported by the Operation Management Laboratory (OML) of the National Institutes of Natural Sciences (NINS), Japan (K.N.).
This paper includes data collected with the TESS mission, obtained from the MAST data archive at the Space Telescope Science Institute (STScI). Funding for the TESS mission is provided by the NASA Explorer Program. STScI is operated by the Association of Universities for Research in Astronomy, Inc., under NASA contract NAS 5-26555.
Some of the data presented in this paper were obtained from the Mikulski Archive for Space Telescopes (MAST) at the Space Telescope Science Institute.
The authors acknowledge ideas from the participants in the workshop ``Blazing Paths to Observing Stellar and Exoplanet Particle Environments" organized by the W.M. Keck Institute for Space Studies.
The authors also would like to acknowledge the the relevant discussions in the International Space Science Institute (ISSI)
Workshop ``Stellar Magnetism and its Impact on (Exo)Planets (\url{https://workshops.issibern.ch/stellar-magnetism/})".
\end{ack}

\begin{comment}
\section*{Funding}
 This research was supported by ...

\section*{Data availability}
 The data underlying this article are available ...

% Sample Data Availability Statements
% https://academic.oup.com/pages/open-research/research-data#Data%20Availability%20Statements

\appendix %%%%%%%%%%%%%%%%%%%%%%%%%%%%%%%%%%%%%%%%%%%%%%%%%%%%%%%%
\section*{Case of single paragraph}
 No section number is necessary. Add ``*'' after \verb/\section/.

%%%%
\section{Case of two or more paragraphs}

 Text of appendix

\section{Case of two or more paragraphs}

 Text of appendix

%\cite{2022NatAs...6..241N}

% Any journal's BST file (e.g., apj.bst) can be used as PASJ's BST is unavailable.

\end{comment}

% \bibliographystyle{****}
% \bibliography{****}
%\bibliography{tess.student.sun-like-star}{}

%bibliographystyle{pasj}
%\bibliographystyle{pasj}
%\bibliographystyle{plainnat}
%\bibliography{tess_student_sun-like-star}

\begin{thebibliography}{}
\bibitem[Airapetian et al.(2016)]{Airapetian_2016} Airapetian, V. S., Glocer, A., Gronoff, G., et al. 2016, Nat. Geosci., 9, 452

\bibitem[Airapetian et al.(2020)]{Airapetian_2020}Airapetian, V. S., Barnes, R., Cohen, O., et al. 2020, IJAsB, 19, 136

\bibitem[Namekata et al.(2022a)]{Namekata_2022a} Namekata, K., Maehara, H., Honda, S., et al. 2022a, NatAs, 6, 241

\bibitem[Namekata et al.(2022b)]{Namekata_2022b} Namekata, K., Maehara, H., Honda, S., et al. 2022b, arXiv:2211.05506.

\bibitem[Newton et al.(2019)]{2019ApJ...880L..17N} Newton, E.~R., Mann, A.~W., Tofflemire, B.~M., et al.\ 2019, \apjl, 880, 1, L17. doi:10.3847/2041-8213/ab2988

\bibitem[Bell et al.(2015)]{2015MNRAS.454..593B} Bell, C.~P.~M., Mamajek, E.~E., \& Naylor, T.\ 2015, \mnras, 454, 1, 593. doi:10.1093/mnras/stv1981

\bibitem[Waite et al.(2017)]{2017MNRAS.465.2076W} Waite, I.~A., Marsden, S.~C., Carter, B.~D., et al.\ 2017, \mnras, 465, 2, 2076. doi:10.1093/mnras/stw2731

\bibitem[Strassmeier et al.(2003)]{2003A&A...411..595S} Strassmeier, K.~G., Pichler, T., Weber, M., et al.\ 2003, \aap, 411, 595. doi:10.1051/0004-6361:20031538





\end{thebibliography}

\end{document}

=== Frequently used abbreviation of journal names ===
\aj         AJ
\araa       ARA\&A
\apj        ApJ
\apjl       ApJL
\apjs       ApJS
\apss       Ap\&SS
\aap        A\&A
\aapr       A\&AR
\aaps       A\&AS
\baas       BAAS
\icarus     ICARUS
\mnras      MNRAS
\prd        Phys.\ Rev.\ D
\prl        Phys.\ Rev.\ Lett.
\pasp       PASP
\pasj       PASJ
\solphys    Sol.\ Phys.
\ssr        Space\ Sci.\ Rev.
\nat        Nature
\iaucirc    IAU\ Circ.
\gca        Geochim.\ Cosmochim.\ Acta
\jgr        J.\ Geophys.\ Res.
\nphysa     Nucl.\ Phys.\ A
\procspie   Proc.\ SPIE
\aip        AIP Conf.\ Proc.
\asp        ASP Conf.\ Ser.
=====================================================
